\documentclass{article}
\usepackage{geometry}
\geometry{a4paper, margin=1in}
\usepackage{amsmath}
\usepackage{graphicx}
\usepackage{hyperref}
\usepackage{ragged2e}

\title{\textbf{ LZIFC005L - Computer Science Ecosystems}}
\author{Alibek Zhumabayev, Efe Kan}
\date{\today}

\begin{document}

\maketitle

\begin{abstract}
\justifying
This paper summarizes the goals and conclusions from our collaborative web design project for the ecosystems class. The project's goal was to create an entertaining and dynamic web page utilizing HTML, CSS, JavaScript, and XML, displaying our expertise in these technologies. Git and GitHub were essential to our process, allowing for rapid collaboration and version control among team members. The document describes the approaches used, the obstacles encountered, and the solutions developed during the project. Finally, our efforts yielded a responsive, user-friendly web page, emphasizing the value of collaboration and technical expertise in effective web development.

\end{abstract}

\section{Introduction}
This document serves as the documentation for the second coursework of the ecosystems class. The project involved creating a web page to demonstrate proficiency in HTML, CSS, JavaScript, and XML. Additionally, Git was utilized for collaborative work, with the entire project being developed in a repository on the GitHub platform.

Throughout the creation of the webpage, Git and GitHub were extensively used by the group members to maintain a professional workflow, enabling the sharing of various versions and updates among the team.

This document will outline the methods employed to complete the project, as well as the challenges faced and the discoveries made during its development.

\section{Method's}
Multiple languages were used to complete this project :
\begin{itemize}
    \item \textbf{Git/GitHub}: GitHub was used in this project because it involved multiple users and a lot of work. First, a common repository was created on GitHub and all work was carried out there. Every change made to the codes was recorded step by step and documented in detail and added to this repository for common access by group members. Thus, team members were able to easily follow all updates made to the code and contribute to the project.
    \item \textbf{HTML/CSS}: Half of this project includes html and css codes. An interactive web page was created with HTML and HTML was used to create the web page and create the outline. CSS is used to create the style, colors, text font and properties of the web page. In addition, flexbox is used in a large part of the web page, flex box ; It ensures that the user can view the website regardless of technological device and that the page is device-responsive, and the web page provides the user with a good user experience.
    \item \textbf{JavaScript fetch}: JavaScript was used to increase user interaction of the web page created with HTML and CSS. When you interact with the "play", "pause" and "message" sections on the web page, you will see some notifications on the console. These notifications provide instant feedback to the user about the actions they take and contribute to making the web page more dynamic and user-friendly.
    \item \textbf{XML}: 
This XML document describes the structure and essential components of the Coursework 2 project for the Ecosystems course. It contains project details, team members, tasks, and milestones. The project information part includes the project's name, objectives, and start and finish dates. The team members section includes a list of all team members, their positions, and contact information. The tasks section lists tasks by ID, name, assignee, status, and due date. The milestones section identifies key milestones with IDs, names, and dates. This XML format promotes efficient project management and cooperation.

\end{itemize}

\section{Findings and Outcomes}

Initially, we as a group had difficulties using GitHub and accessing a shared repository. As a result of our research, we found a way to use common repositories. Secondly, we repeatedly encountered the "this file is not suitable" error when downloading the LaTeX file, and we tried several different browsers to solve this problem. The most important discovery was how we could share a repository on GitHub with our group friends. These experiences helped us overcome the technical problems we encountered during the project and manage teamwork more effectively.
  

\section{Conclusion}

In conclusion, we learned a lot about web development and collaborative technologies through this project. Initially, we had difficulties when using GitHub and accessing a shared repository. However, through research and perseverance, we discovered how to efficiently use common repositories for collaborative projects. Furthermore, we found and fixed problems downloading LaTeX files by experimenting with various browsers. The most important discovery was learning how to share a repository on GitHub with team members. These experiences not only helped us solve technical challenges, but also increased our ability to manage teams effectively. We were able to develop a responsive and interactive web page using HTML, CSS, JavaScript, and XML, which improved our technical skills and laid the groundwork for future projects.Overall, this coursework has been instrumental in advancing our understanding of web technologies and effective collaboration.


\end{document}
